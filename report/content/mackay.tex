\section{MacKay \& Neal LDPC codes}
Mac kay \& Neal codes are a class of code that can be generated with rank and
non-regular degree distribution constraints (contrary to Gallager codes).

%%%%%%%%%%%%%%%%%%%%%%%%%%%%%%%%%%%%%%%%%%%%%%%%%%%%%%%%%%%%%%%%%%%%%%%%%%%%%%%%
% GENERATION                                                                   %
%%%%%%%%%%%%%%%%%%%%%%%%%%%%%%%%%%%%%%%%%%%%%%%%%%%%%%%%%%%%%%%%%%%%%%%%%%%%%%%%
\subsection{Generation}
The generation of such codes is done iteratively on the columns of the parity
check matrix $\bm{H}$. Assuming that the first $j$ columns have been generated,
we randomly draw a column $\bm{\tilde{h}}$ with degree $\bm{d_v}[j+1]$. If the
row degree constraints are satisfied up until the column $j+1$, then
$\bm{h}_{j+1} \leftarrow \bm{\tilde{h}}$, else, retry. To ensure convergence, a
backtracking mechanism is used. A maximum number of retries $n_\text{retry}$ is
set. If it is reach while trying to generate a column $j + 1$, then we erase the
last $t$ columns and restart from $j - t$.

Therefore, there are two parameters to set in this algorithm: the number of
retries $n_\text{retry}$ and the backtracking depth $t$. A rule of thumb for the
number of retries is the following. If we want to create a column with $M$
coefficients and degree $d_v$, there are $\binom{M}{d_v}$ possibilities. The
number of retries should be of the order of this value (which can get pretty
large). However, for large code sizes, this method quickly becomes intractable
(the number of combinations explodes). The backtracking depth is arbitrarily set
to TODO.

%%%%%%%%%%%%%%%%%%%%%%%%%%%%%%%%%%%%%%%%%%%%%%%%%%%%%%%%%%%%%%%%%%%%%%%%%%%%%%%%
% MONTECARLO SIMULATION ON THE BMC CHANNEL                                     %
%%%%%%%%%%%%%%%%%%%%%%%%%%%%%%%%%%%%%%%%%%%%%%%%%%%%%%%%%%%%%%%%%%%%%%%%%%%%%%%%
\subsection{Montecarlo simulation on the BMC channel}

The objective of this part is to study the effect of the irregularity of the
degree distribution and rank of Mac Kay \& Neal codes on their performance in
terms of \acrshort{ber} and \acrshort{bler}.

%%%%%%%%%%%%%%%%%%%%%%%%%%%%%%%%%%%%%%%%%%%%%%%%%%%%%%%%%%%%%%%%%%%%%%%%%%%%%%%%
\subsubsection{Regular Mac Kay \& Neal codes}
First, we consider codes with a regular degree distribution as for Gallager
codes (\autoref{sec:gallager_codes}).

%%%%%%%%%%%%%%%%%%%%%%%%%%%%%%%%%%%%%%%%%%%%%%%%%%%%%%%%%%%%%%%%%%%%%%%%%%%%%%%%
\subsubsection{Irregular Mac Kay \& Neal codes}

%%%%%%%%%%%%%%%%%%%%%%%%%%%%%%%%%%%%%%%%%%%%%%%%%%%%%%%%%%%%%%%%%%%%%%%%%%%%%%%%
\subsubsection{Irregular full rank Mac Kay \& Neal codes}


How to choose the degree distributions ? Regular degree: take the first smallest
values verifying the condition on the total degree. Irregular degree: take the
previous regular degrees and perturb an given proportion of bit node degrees
(i.e. connect a given number of bits to a higher number of check nodes).

Implementation issue: spend a lot of time computing the tanh and atanh (see
flamegraph)